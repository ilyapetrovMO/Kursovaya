\subsection{Взаимодействие элементов}
Клиент работает с вебсайтом приложения, который может работать с базой базой данных
через RESTful WebAPI, что обеспечивает легкость в разработке, развертке, поддержании и
последующем масштабировании системы. Также, такая архитектура позволяет размещать сайт
и WebAPI на отдельных серверах.

\subsection{Структура разбиения данных в базе данных}
В качестве базы данных используется MySQL Server - реляционная база данных основанная на 
языке SQL\par
Хранение данных происходит в таблицах User и Devices.\par
Основные поля таблицы User:
\begin{itemize}
    \item login --- Имя пользователя
    \item email --- адрес электронной почты пользователя
    \item pswdHash --- хешированый пароль
    \item idUser --- уникальный ключ пользователя
    \item siteSetting --- поле для сохранения настроек пользователя
\end{itemize}
Поля таблицы Devices:
\begin{itemize}
    \item idUser --- уникальный ключ пользователя
    \item deviceName --- имя устройства, заданное пользователем
    \item deviceKey --- уникальный ключ устройства
    \item deviceData --- данные, полученные с устройства и сцепленные в одну строку
\end{itemize}

\subsection{Диаграмма прецедентов}
Диаграмма прецедентов отражает отношения между актерами и прецедентами и позволяет 
описать систему на концептуальном уровне. На рисунке представлена диаграмма прецедентов.

\begin{table}[]
    \begin{tabular}{ll}
    Название прецедента       & Регистрация.\\
    Основное действующее лицо & Пользователь.\\
    Цель                      & Зарегистрироваться на сайте.\\
    Триггер                   & Пользователь решает пройти регистрацию и заходит на сайт.\\
    Результат                 & Пользователь создал учетную запись. Данные о ней добавлены в БД.
    \end{tabular}
\end{table}

\section{Основная последовательность}
\begin{itemize}
    \item Пользователь выбирает пункт меню <<Регистрация>>.
    \item Сайт отображает страницу с формой регистрации.
    \item Пользователь заполняет форму регистрации.
    \item Страница отправляет данные методом "Post" на обработку и хранение. 
\end{itemize}

\begin{table}[]
    \begin{tabular}{ll}
    Название прецедента       & Авторизация.\\
    Основное действующее лицо & Пользователь.\\
    Цель                      & Авторизоваться на сайте.\\
    Триггер                   & Пользователь решает зайти в свою учетную запись.\\
    Результат                 & Пользователь заходит в свою учетную запись если предоставлена верная пара логин пароль, иначе происходит отказ авторизации. 
    \end{tabular}
\end{table}

\section{Основная последовательность}
\begin{itemize}
    \item Пользователь выбирает пункт меню <<Авторизация>>.
    \item Сайт отображает страницу с формой авторизации.
    \item Пользователь заполняет форму авторизации.
    \item Страница авторизует пользователя
\end{itemize}
\section{Альтернативная последовательность}
\begin{itemize}
    \item Пользователь заполняет форму авторизации.
    \item Система сообщает об отказе.
\end{itemize}

\begin{table}[]
    \begin{tabular}{ll}
    Название прецедента       & Добавление устройства.\\
    Основное действующее лицо & Пользователь.\\
    Цель                      & Добавить новое устройство для работы с системой.\\
    Триггер                   & Пользователь решает добавить новое устройство.\\
    Результат                 & Пользователь создал новое устройство. Данные добавлены в бд.
    \end{tabular}
\end{table}

\section{Основная последовательность}
\begin{itemize}
    \item Пользователь выбирает пункт меню <<Добавить устройство>>.
    \item Сайт отображает страницу с формой добавления устройства.
    \item Пользователь заполняет форму.
    \item Страница отправляет данные методом "Post" на обработку и хранение. 
\end{itemize}

\begin{table}[]
    \begin{tabular}{ll}
    Название прецедента       & Редактирование устройства.\\
    Основное действующее лицо & Пользователь.\\
    Цель                      & Обновил данные устройства.\\
    Триггер                   & Пользователь решает изменить данные устройства.\\
    Результат                 & Пользователь изменил данные об устройстве. Данные добавлены в бд.
    \end{tabular}
\end{table}

\section{Основная последовательность}
\begin{itemize}
    \item Пользователь выбирает пункт меню <<Редактировать устройство>>.
    \item Сайт отображает страницу с формой информации об устройстве.
    \item Страница получает данные методом "Get" и заполняет форму. 
    \item Пользователь изменяет данные в форме.
    \item Страница отправляет данные методом "Update" на обработку и хранение. 
\end{itemize}

\begin{table}[]
    \begin{tabular}{ll}
    Название прецедента       & Удаление устройства.\\
    Основное действующее лицо & Пользователь.\\
    Цель                      & Удалить устройство из системы.\\
    Триггер                   & Пользователь решает удалить добавленное устройство.\\
    Результат                 & Пользователь удалил ранее добавленное устройство. Данные удалены из бд.
    \end{tabular}
\end{table}

\section{Основная последовательность}
\begin{itemize}
    \item Пользователь выбирает пункт меню <<Удаление устройств>>.
    \item Сайт отображает страницу с формой удаления устройства.
    \item Пользователь выбирает устройство из списка и нажимает на пункт <<Удаление устройства>>.
    \item Страница отправляет данные методом "Delete" на удаление данных об устройстве. 
\end{itemize}

\begin{table}[]
    \begin{tabular}{ll}
    Название прецедента       & Настройка оповещений.\\
    Основное действующее лицо & Пользователь.\\
    Цель                      & Настроить уведомления.\\
    Триггер                   & Пользователь решает настроить оповещения для устройства.\\
    Результат                 & Пользователь настроил оповещения. Данные добавленны бд.
    \end{tabular}
\end{table}

\section{Основная последовательность}
\begin{itemize}
    \item Пользователь выбирает пункт меню <<Настройка оповещений>>.
    \item Сайт отображает страницу с формой настройки оповещений.
    \item Пользователь выбирает устройство из списка, данные устройства и условия триггера.
    \item Пользователь настраивает текст оповещения, выбирает тип оповещения и нажимает пункт <<Добавить оповещение>>. 
    \item Страница отправляет данные методом "Post" на обработку и хранение. 
\end{itemize}
\section{Альтернативная последовательность}
\begin{itemize}
    \item Пользователь выбирает пункт меню <<Настройка оповещений>>.
    \item Сайт отображает страницу с формой настройки оповещений.
    \item Пользователь выбирает устройство из списка и пункт <<Удаление оповещение>>.
    \item Страница отправляет данные методом "Delete" на удаление данных об оповещение. 
\end{itemize}

\section{Основная последовательность}
\begin{itemize}
    \item Пользователь выбирает пункт меню <<Удаление устройств>>.
    \item Сайт отображает страницу с формой удаления устройства.
    \item Пользователь выбирает устройство из списка и нажимает на пункт <<Удаление устройства>>.
    \item Страница отправляет данные методом "Delete" на удаление данных об устройстве. 
\end{itemize}

\begin{table}[]
    \begin{tabular}{ll}
    Название прецедента       & Настройка вида.\\
    Основное действующее лицо & Пользователь.\\
    Цель                      & Настроить вид отображения данных.\\
    Триггер                   & Пользователь решает настроить вид отображения данных для устройства.\\
    Результат                 & Пользователь настроил вид. Данные добавленны бд.
    \end{tabular}
\end{table}

\section{Основная последовательность}
\begin{itemize}
    \item Пользователь выбирает пункт меню <<Настройка вида>>.
    \item Сайт отображает страницу с формой настройки вида.
    \item Пользователь выбирает устройство из списка.
    \item Пользователь настраивает вид и отображаемые данные, и выбирает пункт <<Сохранить вид>>. 
    \item Страница отправляет данные методом "Post" на обработку и хранение. 
\end{itemize}

\begin{table}[]
    \begin{tabular}{ll}
    Название прецедента       & Просмотр данных в базе.\\
    Основное действующее лицо & Пользователь.\\
    Цель                      & Посмотреть данные в исходном виде.\\
    Триггер                   & Пользователь решает посмотреть данные с устройства.\\
    Результат                 & Пользователь увидел данные с устройства.
    \end{tabular}
\end{table}

\section{Основная последовательность}
\begin{itemize}
    \item Пользователь выбирает пункт меню <<Просмотр данных в базе>>.
    \item Сайт отображает страницу с формой выбора устройства.
    \item Пользователь выбирает устройство из списка.
    \item Страница данные данные методом "Get" и отображает их в исходном виде. 
\end{itemize}

\begin{table}[]
    \begin{tabular}{ll}
    Название прецедента       & Просмотр данных в виде.\\
    Основное действующее лицо & Пользователь.\\
    Цель                      & Посмотреть данные в настроенном виде.\\
    Триггер                   & Пользователь решает посмотреть данные с устройства.\\
    Результат                 & Пользователь увидел данные с устройства.
    \end{tabular}
\end{table}

\section{Основная последовательность}
\begin{itemize}
    \item Пользователь выбирает пункт меню <<Просмотр данных в виде>>.
    \item Сайт отображает страницу с формой выбора устройства.
    \item Пользователь выбирает устройство из списка.
    \item Страница данные данные методом "Get" и отображает их в настроенном виде. 
\end{itemize}

\begin{table}[]
    \begin{tabular}{ll}
    Название прецедента       & Редактирование данных устройства.\\
    Основное действующее лицо & Пользователь.\\
    Цель                      & Обновил данные в базе.\\
    Триггер                   & Пользователь решает изменить данные в базе.\\
    Результат                 & Пользователь изменил данные с устройства. Данные обновленны в бд.
    \end{tabular}
\end{table}

\section{Основная последовательность}
\begin{itemize}
    \item Пользователь выбирает пункт меню <<Редактировать данные>>.
    \item Сайт отображает страницу с формой информации с устройства.
    \item Страница получает данные методом "Get" и заполняет форму. 
    \item Пользователь изменяет данные в форме.
    \item Страница отправляет данные методом "Update" или "Delete" на обработку и хранение. 
\end{itemize}

\begin{table}[]
    \begin{tabular}{ll}
    Название прецедента       & Авторизация как администратор.\\
    Основное действующее лицо & Администратор.\\
    Цель                      & Авторизоваться на сайте как администратор.\\
    Триггер                   & Администратор решает зайти в свою учетную запись.\\
    Результат                 & Администратор заходит в свою учетную запись если предоставлена верная пара логин пароль, иначе происходит отказ авторизации. 
    \end{tabular}
\end{table}

\section{Основная последовательность}
\begin{itemize}
    \item Администратор выбирает пункт меню <<Авторизация как администратор>>.
    \item Сайт отображает страницу с формой авторизации.
    \item Администратор заполняет форму авторизации.
    \item Страница авторизует администратора
\end{itemize}
\section{Альтернативная последовательность}
\begin{itemize}
    \item Администратор заполняет форму авторизации.
    \item Система сообщает об отказе.
\end{itemize}

\begin{table}[]
    \begin{tabular}{ll}
    Название прецедента       & Отправка данных.\\
    Основное действующее лицо & IoT девайс.\\
    Цель                      & Отправить данные в систему.\\
    Триггер                   & Прошло время между отправками пакета данных.\\
    Результат                 & Данные с устройства записались в базу данных. 
    \end{tabular}
\end{table}

\section{Основная последовательность}
\begin{itemize}
    \item Устройство записывает данные в JSON файл.
    \item Устройство отправляет файл на сервер.
    \item Файл обрабатывается на сервере.
    \item Данные из файла записываются в базу данных.
\end{itemize}

\begin{table}[]
    \begin{tabular}{ll}
    Название прецедента       & Прием данных.\\
    Основное действующее лицо & IoT девайс.\\
    Цель                      & Принять данные из систему.\\
    Триггер                   & Пользователь решает оправить данные на девайс.\\
    Результат                 & Данные успешно обработанны устройством. 
    \end{tabular}
\end{table}

\section{Основная последовательность}
\begin{itemize}
    \item Пользователь отправляет данные на сервер.
    \item Сервер обрабатывает данные и отправляет их устройству.
    \item Устройство принимает данные и обрабатывает их.
\end{itemize}

