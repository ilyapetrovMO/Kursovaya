\subsection{Взаимодействие элементов}
Клиент работает с вебсайтом приложения, который может работать с базой базой данных
через RESTful WebAPI, что обеспечивает легкость в разработке, развертке, поддержании и
последующем масштабировании системы. Также, такая архитектура позволяет размещать сайт
и WebAPI на отдельных серверах.

\subsection{Структура разбиения данных в базе данных}
В качестве базы данных используется MySQL Server - реляционная база данных основанная на 
языке SQL\par
Хранение данных происходит в таблицах User и Devices.\par
Основные поля таблицы User:
\begin{itemize}
    \item login --- Имя пользователя
    \item email --- адрес электронной почты пользователя
    \item pswdHash --- хешированый пароль
    \item idUser --- уникальный ключ пользователя
    \item siteSetting --- поле для сохранения настроек пользователя
\end{itemize}
Поля таблицы Devices:
\begin{itemize}
    \item idUser --- уникальный ключ пользователя
    \item deviceName --- имя устройства, заданное пользователем
    \item deviceKey --- уникальный ключ устройства
    \item deviceData --- данные, полученные с устройства и сцепленные в одну строку
\end{itemize}

\subsection{Диаграмма прецедентов}
Диаграмма прецедентов отражает отношения между актерами и прецедентами и позволяет 
описать систему на концептуальном уровне. На рисунке представлена диаграмма прецедентов.

\begin{table}[]
    \begin{tabular}{ll}
    Название прецедента       & Регистрация.\\
    Основное действующее лицо & Пользователь.\\
    Цель                      & Зарегистрироваться на сайте.\\
    Триггер                   & Пользователь решает пройти регистрацию и заходит на сайт.\\
    Результат                 & Пользователь создал учетную запись. Данные о ней добавлены в БД.
    \end{tabular}
\end{table}

\section{Основная последовательность}
\begin{itemize}
    \item Пользователь выбирает пункт меню <<Регистрация>>.
    \item Сайт отображает страницу с формой регистрации.
    \item Пользователь заполняет форму регистрации.
    \item Страница отправляет данные методом "Post" на обработку и хранение. 
\end{itemize}

\begin{table}[]
    \begin{tabular}{ll}
    Название прецедента       & Авторизация.\\
    Основное действующее лицо & Пользователь.\\
    Цель                      & Авторизоваться на сайте.\\
    Триггер                   & Пользователь решает зайти в свою учетную запись.\\
    Результат                 & Пользователь заходит в свою учетную запись если предоставлена верная пара логин пароль, иначе происходит отказ авторизации. 
    \end{tabular}
\end{table}

\section{Основная последовательность}
\begin{itemize}
    \item Пользователь выбирает пункт меню <<Авторизация>>.
    \item Сайт отображает страницу с формой авторизации.
    \item Пользователь заполняет форму авторизации.
    \item Страница авторизует пользователя
\end{itemize}
\section{Альтернативная последовательность}
\begin{itemize}
    \item Пользователь заполняет форму авторизации.
    \item Система сообщает об отказе.
\end{itemize}

\begin{table}[]
    \begin{tabular}{ll}
    Название прецедента       & Добавление устройства.\\
    Основное действующее лицо & Пользователь.\\
    Цель                      & Добавить новое устройство для работы с системой.\\
    Триггер                   & Пользователь решает добавить новое устройство.\\
    Результат                 & Пользователь создал новое устройство. Данные добавлены в бд.
    \end{tabular}
\end{table}

\section{Основная последовательность}
\begin{itemize}
    \item Пользователь выбирает пункт меню <<Добавить устройство>>.
    \item Сайт отображает страницу с формой добавления устройства.
    \item Пользователь заполняет форму.
    \item Страница отправляет данные методом "Post" на обработку и хранение. 
\end{itemize}